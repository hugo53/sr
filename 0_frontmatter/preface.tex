%-------Preface
%
\clearpage
\thispagestyle{empty}
%\addcontentsline{toc}{chapter}{Lời nói đầu}
%Note:
% - xem xet  lại tu mot so phuong pháp trích xuất sự kiện tiêu biểu (muốn thay
% bằng một con số cụ thể)
%
\begin{preface}
	\bigskip
    \noindent Được cộng đồng nghiên cứu khoa học trên toàn thế giới quan tâm rất
    sớm, trích xuất sự kiện được xem là một bài toán quan trọng trong lĩnh vực
    trích chọn thông tin (Information Extraction). Từ năm 1987, trích xuất sự
    kiện đã trở thành đề tài chủ chốt tại hội nghị \emph{Message Understanding
    Conferences} ngay lần tổ
	chức đầu tiên (MUC--1) \cite{RB96}. Từ đó đến nay, nhiều phương pháp trích
    xuất sự kiện đã được đưa ra  và áp dụng trong các hệ thống thực tế như
    BioCaster (\href{http://born.nii.ac.jp/}{http://born.nii.ac.jp/}), HealthMap
    (\href{http://healthmap.org}{http://healthmap.org}), EpiSpider\\
    (\href{http://www.epispider.org/}{http://www.epispider.org/}), Metro
    Monitor (\href{ http://www.metromonitor.com/}{
    http://www.metromonitor.com/}), \ldots \\
   \noindent Công trình nghiên cứu \textbf{Một phương pháp lai trích xuất sự kiện và áp
   dụng vào hệ thống theo dõi tin tức trực tuyến  $\mathcal{N}$\texttt{ewSOMoni}} khảo sát một số phương pháp trích xuất sự
    kiện tiêu biểu có hiệu quả tốt, đang được sử dụng trong nhiều hệ thống
    theo dõi thông tin. Dựa trên cơ sở đó, chúng tôi  nghiên cứu và đề xuất một phương
    pháp lai nhằm mục đích trích xuất sự kiện trên miền tin tức tiếng Việt và
    thử nghiệm trên hệ thống theo dõi tin tức trực tuyến
    $\mathcal{N}$\texttt{ewSOMoni}. Phương pháp được đề xuất là sự kết hợp của
    phương pháp học máy Maximum Entropy và phương pháp trích xuất dựa trên
    luật với những cải tiến khi áp dụng cho dữ liệu tiếng Việt. Qua tiến hành
    thực nghiệm, chúng tôi đã thu được kết quả tương đối tốt và ổn định. Điều
    này chứng tỏ tính đúng đắn của phương pháp đề xuất cũng như tính thực tiễn
    trong hệ thống theo dõi tin tức trực tuyến, góp phần đưa thông tin đến với
    người dùng chính xác, kịp thời.
    \\[0.5cm]
	\noindent Báo cáo bao gồm bốn chương được mô tả như dưới đây.

	\begin{description}
		\item[Chương 1.] \emph{Giới thiệu} khái quát chung về động lực thực hiện
		nghiên cứu, mô tả về bài toán trích xuất sự kiện và cũng nêu một số
		nghiên cứu liên quan ở trong và ngoài nước. 
	%Ngoài ra, một hệ thống theo dõi tin tức cũng được nhắc tới trong chương này.


        \item[Chương 2.] \emph{Phương pháp trích xuất sự kiện} đưa ra
        3 phương pháp trích xuất sự kiện phổ biến và có độ chính xác cao. Hơn
        nữa, chúng tôi cũng phân tích những thuận lợi của từng phương pháp và
        cách áp dụng chúng vào mô hình giải quyết của mình để đạt được hiệu quả
        tốt hơn.

        \item[Chương 3.] \emph{Trích xuất sự kiện dựa trên luật kết hợp học máy
		và hệ thống theo dõi tin tức} trình bày phương pháp trích xuất sự kiện
        dựa trên luật kết hợp với phương pháp học máy Maximum Entropy--phương
        pháp chính trong mô hình giải quyết của nghiên cứu này. Đồng thời, mô
		hình hệ thống theo dõi tin tức cũng sẽ được nêu rõ và phân tích chi
		tiết.

\newpage
\thispagestyle{empty}
        \item[Chương 4.] \emph{Thực nghiệm phương pháp trên hệ thống theo dõi
        tin tức} trình bày quá trình xây dựng hệ thống giám sát tin tức trên cơ
        sở  áp dụng phương pháp đã đề xuất ở Chương 3. Kết quả thực nghiệm và
        đánh giá hiệu quả sẽ được mô tả kỹ lưỡng trong chương này.


		\item[Phần kết luận] tổng kết, tóm lược nội dung của nghiên cứu
	    và hướng phát triển tiếp theo.
	\end{description}

	%\indent Cuối cùng, phần Phụ lục trình bày một số nội dung cài đặt của tác giả khóa luận.
\end{preface}
%\endinput
