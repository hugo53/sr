
% this file is called up by thesis.tex
% content in this file will be fed into the main document

%: ----------------------- introduction file header -----------------------
\chapter{Giới thiệu}

% the code below specifies where the figures are stored
\ifpdf
    \graphicspath{{1_introduction/figures/PNG/}{1_introduction/figures/PDF/}{1_introduction/figures/}}
\else
    \graphicspath{{1_introduction/figures/EPS/}{1_introduction/figures/}}
\fi

% ----------------------------------------------------------------------
%: ----------------------- introduction content -----------------------
% ----------------------------------------------------------------------


%: ----------------------- HELP: latex document organisation
% the commands below help you to subdivide and organise your thesis
%    \chapter{}       = level 1, top level
%    \section{}       = level 2
%    \subsection{}    = level 3
%    \subsubsection{} = level 4
% note that everything after the percentage sign is hidden from output




%: ----------------------- HELP: special characters
% above you can see how special characters are coded; e.g. $\alpha$
% below are the most frequently used codes:
%$\alpha$  $\beta$  $\gamma$  $\delta$

%$^{chars to be superscripted}$  OR $^x$ (for a single character)
%$_{chars to be suberscripted}$  OR $_x$

%>  $>$  greater,  <  $<$  less
%≥  $\ge$  greater than or equal, ≤  $\ge$  lesser than or equal
%~  $\sim$  similar to

%$^{\circ}$C   ° as in degree C
%±  \pm     plus/minus sign

%$\AA$     produces  Å (Angstrom)


%: ----------------------- HELP: references
% References can be links to figures, tables, sections, or references.
% For figures, tables, and text you define the target of the link with %\label{XYZ}. Then you call cross-link with the command \ref{XYZ}, as above
% Citations are bound in a very similar way with \cite{XYZ}. You store your %references in a BibTex file with a programme like BibDesk.





%\figuremacro{Ten file anh, dat trong figures}{A common glucose polymers}{The %figure shows starch %granules in potato cells, taken from %\href{http://molecularexpressions.com/micro/gallery/burgersnfries/burgersnfries4.html}{Molecular %Expressions}.}

%: ----------------------- HELP: adding figures with macros
% This template provides a very convenient way to add figures with minimal code.
% \figuremacro{1}{2}{3}{4} calls up a series of commands formating your image.
% 1 = name of the file without extension; PNG, JPEG is ok; GIF doesn't work
% 2 = title of the figure AND the name of the label for cross-linking
% 3 = caption text for the figure

%: ----------------------- HELP: www links
% You can also see above how, www links are placed
% \href{http://www.something.net}{link text}

%\figuremacroW{largepotato}{Title}{Caption}{0.8}
% variation of the above macro with a width setting
% \figuremacroW{1}{2}{3}{4}
% 1-3 as above
% 4 = size relative to text width which is 1; use this to reduce figures

%Intro here
\noindent Thế giới đang thay đổi rất mạnh  với sự tham gia của các phương tiện truyền thông xã hội. Mọi thông tin tới với người dùng nhanh, từ nhiều nguồn khác nhau. Để đáp ứng nhu cầu đó, những hệ thống tổng hợp tin tức lần lượt ra đời giúp cho con người có thể dễ dàng nắm bắt thông tin. Khởi đầu bởi <tên hệ thóng đầu tiên trên thế giới>, tiếp sau đó là  <tên hệ thống khác 1>, <tên hệ thống khác 2>, \ldots Vào năm 2005, hệ thống tổng hợp tin tức tự động đầu tiên của Việt Nam ra đời dựa trên thành tựu nghiên cứu \emph{Hệ thống thu thập và tách thông tin ICPS} của hai tác giả Nguyễn Thành Long và Nguyễn Phú Bình đạt giải nhì cuộc thi Trí Tuệ Việt Nam 2002. Hệ thống có tên \emph{Hệ thống xử lý tiếng Việt tự động ePi} này được đặt tại \href{www.baomoi.com}{www.baomoi.com} được người dùng biết đến với tên \textsc{Báo mới}

\section{Động lực nghiên cứu}


\section{Vấn đề nghiên cứu}
	\subsection{Bài toán}
    \subsection{Câu hỏi nghiên cứu}
    \subsection{Thách thức}

\section{Ý nghĩa}
    \subsection{Ý nghĩa khoa học}

    \subsection{Ý nghĩa thực tiễn}





\section{Nghiên cứu liên quan}
	\subsection{Một số nghiên cứu liên quan ở nước ngoài}

	\subsection{Một số nghiên cứu liên quan ở trong nước}

% ----------------------------------------------------------------------